% !TEX program = xelatex % (can also use "lualatex")

\documentclass{beamer}
\mode<presentation>
{
  \usepackage{theme/theme}
  \setbeamercovered{transparent}
}

\usepackage{amsmath,amssymb,amsfonts}
\usepackage{times}
\usepackage{graphicx}
\usepackage{fancyvrb}
\usepackage{array}
\usepackage{colortbl}
\usepackage{tabularx}
\usepackage{fontspec}
\usepackage{minted}

% Uncomment me when you need to insert code
\usepackage{color}
\usepackage{listings}
% End Code

% End Header

% Titlepage
\title{This is the title of our presentation}
\subtitle{Sometimes we also need a subtitle}
\author{Enterprise Software Architectures}
\institute
{
  Bachelor's Degree in Computer Engineering
}
\date{Academic year 2025/26}
% End Titlepage

% Slides
\begin{document}

% 1. Slide: Titlepage
\begin{frame}
  \titlepage
\end{frame}

% 2. Slide: TOC
\begin{frame}
  \frametitle{Table of contents}
  \tableofcontents[subsectionstyle=hide]
\end{frame}

% Further Slides
\section{Titles}
\begin{frame}
  \frametitle{Title}
  Each frame should have a title.
\end{frame}

\subsection{Subsection no.1.1  }
\begin{frame}
  Without title something is missing.
\end{frame}

\section{Lists}
\subsection{Lists I}
\begin{frame}
  \frametitle{Unnumbered Lists}
  \begin{itemize}
    \item Introduction to  \LaTeX
    \item Second bullet point
    \item And a third one
    \item The last one
  \end{itemize}
\end{frame}

\begin{frame}\frametitle{Lists with Pause}
  \begin{itemize}
    \item Introduction to  \LaTeX \pause
    \item Second bullet point \pause
    \item And a third one \pause
    \item The last one
  \end{itemize}
\end{frame}

\subsection{Lists II}
\begin{frame}
  \frametitle{Numbered Lists}
  \begin{enumerate}
    \item Introduction to  \LaTeX
    \item Second bullet point
    \item And a third one
    \item The last one
  \end{enumerate}
\end{frame}

\begin{frame}
  \frametitle{Numbered Lists with Pause}
  \begin{enumerate}
    \item Introduction to  \LaTeX \pause
    \item Second bullet point \pause
    \item And a third one \pause
    \item The last one
  \end{enumerate}
\end{frame}

\section{Tables}
\subsection{Tables}
\begin{frame}
  \frametitle{Tables}
  \begin{tabular}{|c|c|c|}
    \hline
    \textbf{Title} & \textbf{Title} & \textbf{Title} \\
    \hline
    First column & Second column &  \LaTeX  \\
    \hline
    Second row & \LaTeX & Last column \\
    \hline
  \end{tabular}
\end{frame}

\section{Blocks \& Math}
\subsection{Blocks}
\begin{frame}
  \frametitle{Blocks}

  \begin{block}{This is a simple block}
    It should contain some text.
  \end{block}

  \begin{exampleblock}{Example Block}
    This may be an example.
  \end{exampleblock}

  \begin{alertblock}{Warning}
    The violent color indicates that this block may alert of something.
  \end{alertblock}
\end{frame}

\subsection{Math}
\begin{frame}
  \frametitle{Math Expressions are a Breeze with \LaTeX \ldots}

  \begin{equation}
    p(\mathbf{x}_{k}|\mathbf{Z}_{k}) = \frac{p(\mathbf{z}_{k}|\mathbf{x     }_{k})p(\mathbf{x}_{k}|\mathbf{Z}_{k-1})}{\int \! p(\mathbf{z}_{k}|\mathbf{     x}_{k})p(\mathbf{x}_{k}|\mathbf{Z}_{k-1})\,d\mathbf{x}_{k}}
  \end{equation}

  \begin{equation}
    w_{k}^{i} \sim w_{k-1}^{i}\frac{p(\mathbf{z}_{k}|\mathbf{x}_{k}^{i})p(\mathbf{x}_{k}^{i}|\mathbf{x}_{k-1}^{i})}{q(\mathbf{x}_{k}^{i}|\mathbf{x}_{k-1}^{i},\mathbf{z}_{k})}
  \end{equation}

  \ldots and Bayes filtering is great!

\end{frame}

\section{Multimedia}
\subsection{Split Screen}
\begin{frame}
  \frametitle{Splitting Screen}
  \begin{columns}

    \begin{column}{5cm}
      \begin{itemize}
        \item Here
        \item is some
        \item text
      \end{itemize}
    \end{column}

    \begin{column}{5cm}
      \begin{tabular}{|c|c|}
        \hline
        \textbf{On the} & \textbf{other side} \\
        \hline
        there may &  be a table \\
        \hline
        or even &  a picture as  \\
        \hline
        shown on the &  next frame  \\
        \hline
      \end{tabular}
    \end{column}

  \end{columns}
\end{frame}

\subsection{Pictures}
\begin{frame}
  \frametitle{Pictures in Latex Beamer Class}
  \begin{figure}
    \includegraphics[scale=0.7]{theme/header/campus-igualada.pdf}
    \caption{This is a picture!}
  \end{figure}
\end{frame}

\subsection{Code Listings}
% The [fragile] is important for listings!
\begin{frame}[fragile]
  \frametitle{Code example}
  \lstset{language=C, basicstyle=\small \ttfamily, showspaces=false, showtabs=false, tab= , keywordstyle=\bfseries, showstringspaces=false, framexleftmargin=5mm, frame=single, numbers=left, numberstyle=\tiny, stepnumber=1, numbersep=5pt, texcl=true}
   \begin{minted}{c}
#include <stdio.h>

int main() {
  printf("Hello World\n");
  return 0;
}
   \end{minted}
\end{frame}

% End Slides

\end{document}
