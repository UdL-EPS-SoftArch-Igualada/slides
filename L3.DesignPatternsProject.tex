\documentclass{beamer}
\mode<presentation>
{
  \usepackage{theme/theme}
  \setbeamercovered{transparent}
}

\usepackage{amsmath,amssymb,amsfonts}
\usepackage{times}
\usepackage{graphicx}
\usepackage{fancyvrb}
\usepackage{array}
\usepackage{colortbl}
\usepackage{tabularx}
\usepackage{fontspec}
\usepackage{minted}
\usepackage{libs/tikz-uml}

% Uncomment me when you need to insert code
\usepackage{color}
\usepackage{listings}
% End Code

% End Header

% Titlepage
\newcommand{\maintitle}{L3. Design Patterns Project}
\newcommand{\deliverableDay}{March 15 (Sunday)}
\newcommand{\presentationDay}{March 16 (Monday)}
\title{\maintitle}
\author{Enterprise Software Architectures}
\institute
{
  Bachelor's Degree in Computer Engineering
}
\date{Academic year 2025/26}
% End Titlepage

\AtBeginSection[]{
  \begin{frame}
    \centering
    \begin{beamercolorbox}[sep=8pt,center]{title}
      \usebeamerfont{title}\insertsectionhead
    \end{beamercolorbox}
  \end{frame}
}

% Slides
\begin{document}

\begin{frame}
  \titlepage
\end{frame}

\begin{frame}
  \frametitle{Goals}

  You will be assigned a design pattern at random. You must:
  \begin{itemize}
    \item Do your own research and learn about the pattern.
    \item Create a presentation about your design pattern.
    \item Present the design pattern to the class.
    \begin{itemize}
      \item You will have \textbf{10 minutes}.
      \item Both members of the group \textbf{must} participate in the presentation.
      \item Reading from written notes is \textbf{not} allowed.
      \item You must \textit{understand} what you are talking about, not merely repeat a script.
    \end{itemize}
    \item Answer our questions to demonstrate your understanding.
  \end{itemize}
\end{frame}

\begin{frame}
  \frametitle{Deadlines}

  You must deliver a PDF version of your presentation \href{https://cv.udl.cat/portal}{through the CV} before \textbf{\deliverableDay} at 23:55h.
  \newline
  
  Everyone will present on \textbf{\presentationDay}, in a random order.
\end{frame}

\begin{frame}
  \frametitle{Content of the Presentation (all patterns \textbf{except Singleton})}

  \textbf{Part A: Explain the pattern} (recommended: $\sim$7 minutes\footnote{Just a suggestion, you can distribute your 10 minutes in any way you prefer.})
  \begin{itemize}
    \item See "Parts of a design pattern" in T3.
  \end{itemize}

  \textbf{Part B: Usage in our project} (recommended: $\sim$3 minutes)
  \begin{itemize}
    \item If we use it in our project, explain where and how. Show real code snippets.
    \item If we don't use it (and it would make sense to add it), submit a PR and explain it in your presentation.
    \item If using it doesn't currently make sense, invent a hypothetical new feature \textbf{in our project} where it can be used. Include \textit{correct} Java + Spring Boot code implementing the pattern.
  \end{itemize}
\end{frame}

\begin{frame}
  \frametitle{Content of the Presentation (\textbf{Singleton})}

  \textbf{Part A: Explain the pattern} (recommended: $\sim$5 minutes\footnote{Just a suggestion, you can distribute your 10 minutes in any way you prefer.})
  \begin{itemize}
    \item See "Parts of a design pattern" in T3.
  \end{itemize}

  \textbf{Part B: Alternatives} (recommended: $\sim$5 minutes)
  \begin{itemize}
    \item What is an "Anti-pattern"? Why is Singleton currently considered an anti-pattern?
    \item Explain 1 or 2 ways of accomplishing the same goal as Singleton, without its drawbacks.
  \end{itemize}
\end{frame}

\begin{frame}
  \frametitle{Distributing the patterns}

  \centering
  \renewcommand{\arraystretch}{4}
  \begin{tabular}{ccccc}
    \begin{minipage}{0.16\linewidth}\centering
      \textbf{1}
      \includegraphics[width=\linewidth]{images/T3/patterns/abstract-factory.png}\\
      {\small Abstract Factory}
    \end{minipage}
    &
    \begin{minipage}{0.16\linewidth}\centering
      \textbf{2}
      \includegraphics[width=\linewidth]{images/T3/patterns/builder.png}\\
      {\small Builder}
    \end{minipage}
    &
    \begin{minipage}{0.16\linewidth}\centering
      \textbf{3}
      \includegraphics[width=\linewidth]{images/T3/patterns/decorator.png}\\
      {\small Decorator}
    \end{minipage}
    &
    \begin{minipage}{0.16\linewidth}\centering
      \textbf{4}
      \includegraphics[width=\linewidth]{images/T3/patterns/template-method.png}\\
      {\small Template Method}
    \end{minipage}
    &
    \begin{minipage}{0.16\linewidth}\centering
      \textbf{5}
      \includegraphics[width=\linewidth]{images/T3/patterns/strategy.png}\\
      {\small Strategy}
    \end{minipage}
    \\
    \begin{minipage}{0.16\linewidth}\centering
      \textbf{6}
      \includegraphics[width=\linewidth]{images/T3/patterns/singleton.png}\\
      {\small Singleton}
    \end{minipage}
    &
    \begin{minipage}{0.16\linewidth}\centering
      \textbf{7}
      \includegraphics[width=\linewidth]{images/T3/patterns/proxy.png}\\
      {\small Proxy}
    \end{minipage}
    &
    \begin{minipage}{0.16\linewidth}\centering
      \textbf{8}
      \includegraphics[width=\linewidth]{images/T3/patterns/adapter.png}\\
      {\small Adapter}
    \end{minipage}
    &
    \begin{minipage}{0.16\linewidth}\centering
      \textbf{9}
      \includegraphics[width=\linewidth]{images/T3/patterns/observer.png}\\
      {\small Observer}
    \end{minipage}
    &
    \begin{minipage}{0.16\linewidth}\centering
      \textbf{10}
      \includegraphics[width=\linewidth]{images/T3/patterns/state.png}\\
      {\small State}
    \end{minipage}
  \end{tabular}
\end{frame}


\begin{frame}
  \frametitle{Closing words}

  Both mid-term \textbf{exams may contain questions} about the design patterns presented in class.  \newline

  Remember to stick to \textbf{10 minutes} for your presentation, to allow all your classmates to present in a single day.

  We know it's not easy to condense everything into so little time, but this demonstrates that you have really understood the material.
  \vfill

  \textit{“I apologize for such a long letter - I didn't have time to write a short one.”}

  {\raggedleft
    \footnotesize
    - Mark Twain
    \par}
\end{frame}

\end{document}
