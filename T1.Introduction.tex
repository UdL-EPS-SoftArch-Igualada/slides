\documentclass{beamer}
\mode<presentation>
{
  \usepackage{theme/theme}
  \setbeamercovered{transparent}
}

\usepackage{amsmath,amssymb,amsfonts}
\usepackage{times}
\usepackage{graphicx}
\usepackage{fancyvrb}
\usepackage{array}
\usepackage{colortbl}
\usepackage{tabularx}
\usepackage{fontspec}
\usepackage{minted}

% Uncomment me when you need to insert code
\usepackage{color}
\usepackage{listings}
% End Code

% End Header

% Titlepage
\newcommand{\maintitle}{T1. Introduction}
\title{\maintitle}
\author{Enterprise Software Architectures}
\institute
{
  Bachelor's Degree in Computer Engineering
}
\date{Academic year 2025/26}
% End Titlepage

\AtBeginSection[]{
  \begin{frame}
    \centering
    \begin{beamercolorbox}[sep=8pt,center]{title}
      \usebeamerfont{title}\insertsectionhead
    \end{beamercolorbox}
  \end{frame}
}

% Slides
\begin{document}

\begin{frame}
  \titlepage
\end{frame}

\begin{frame}
  \frametitle{\maintitle}
  \tableofcontents[subsectionstyle=show]
\end{frame}

\section{Block A - Overview of the module}
\subsection{Lecturers}
\begin{frame}
  \frametitle{A - Presentation | Lecturers}
  \begin{itemize}
    \item \textbf{Pol Rivero} (pol.rivero@udl.cat)
      \begin{itemize}
        \item Main lecturer of this module (theory and laboratories) \vfill
      \end{itemize}
    \item With the support of \textbf{Sergio Sayago} (sergio.sayago@udl.cat)
      \begin{itemize}
        \item Session 1
        \item Session 2
        \item Participation in the presentation of the project I (patterns)
      \end{itemize}
  \end{itemize}
\end{frame}

\subsection{Dynamics}
\begin{frame}
  \frametitle{A - Presentation | Dynamics (I) What will we do?}
  \begin{itemize}
    \item \textbf{Highly practical} module
      \begin{itemize}
        \item 75\% of the sessions are laboratories
        \item Java + React + GitHub
        \item Agile Software Development
        \item Intro to: Test-Driven Development + Behaviour-Driven Development
      \end{itemize}
    \item Project-Based Learning
      \begin{itemize}
        \item \textbf{All students} will work on a \textbf{common project} (domain: First Lego League)
        \item Pol will oversee the project and become a guide on your side
      \end{itemize}
  \end{itemize}
\end{frame}

\begin{frame}
  \frametitle{A - Presentation | Dynamics (II) Language of instruction}
  \begin{itemize}
    \item All the \textbf{materials} and \textbf{written communication} will be in \textbf{English}
      \begin{itemize}
        \item We will simulate we are working on an international project
      \end{itemize}
    \item \textbf{Oral communication} in class with Pol will be in \textbf{Catalan}
      \begin{itemize}
        \item In the project, we are one of the partners, which is based in Igualada
      \end{itemize}
    \item Sessions \textbf{1} and \textbf{2} will be \textbf{100\% in English​}
  \end{itemize}
\end{frame}

\subsection{GenAI}
\begin{frame}
  \frametitle{A - Presentation | Generative AI}
  \begin{itemize}
    \item You can use it
    \item Important topics:
      \begin{itemize}
        \item Everyone must understand and be able to explain how their code works and what each line does
        \item \textit{Vibe coding}\footnote{Unlike traditional AI-assisted coding, the human developer avoids examination of the code, accepts AI-suggested completions without human review, and focuses more on iterative experimentation than code correctness or structure (Source: \url{https://en.wikipedia.org/wiki/Vibe_coding})} will \textbf{not} be tolerated
      \end{itemize}
  \end{itemize}
\end{frame}

\subsection{Assessment}
\begin{frame}
  \frametitle{A - Presentation | Assessment - Detailing the Course Guide}
  \begin{itemize}
    \item \textbf{Two written exams}
      \begin{itemize}
        \item 1st midterm exam = 20\%
        \item 2nd midterm exam = 20\%
        \item Written, individual and easy if you follow the module
      \end{itemize}

    \item \textbf{Project}
      \begin{itemize}
        \item 1st deliverable = 20\% \newline (work done: 5\%; project monitoring: 5\%; presentation: 10\%)
        \item 2nd deliverable = 20\% \newline (work done: 15\%; project monitoring: 5\%)
        \item 3rd deliverable = 20\% \newline (work done: 15\%; project monitoring: 5\%)
      \end{itemize}

    \item \textbf{There are no retakes}

  \end{itemize}
\end{frame}

\section{Block B - Getting Started}
\subsection{Task 0: Setting up the groups}
\begin{frame}
  \frametitle{Block B - Getting started | Task 0: Setting up the groups}
  You will work in pairs. 20 students / 2 = 10 groups

  \newcounter{group}

  \begin{columns}[T,onlytextwidth]
    \column{0.48\textwidth}
    \begin{itemize}
        \setcounter{group}{0}
        \loop
        \stepcounter{group}
      \item \textbf{Group \arabic{group}}
        \begin{itemize}
          \item TBC
        \end{itemize}
        \ifnum\value{group}<5
        \repeat
    \end{itemize}

    \column{0.48\textwidth}
    \begin{itemize}
        \setcounter{group}{5}
        \loop
        \stepcounter{group}
      \item \textbf{Group \arabic{group}}
        \begin{itemize}
          \item TBC
        \end{itemize}
        \ifnum\value{group}<10
        \repeat
    \end{itemize}
  \end{columns}
\end{frame}

\subsection{Domain: First Lego League}
\begin{frame}
  \frametitle{Block B - Domain | The First Lego League}
  The domain of the project is the \textbf{First Lego League}
  \begin{itemize}
    \item Spanish chapter: \url{https://firstlegoleague.soy}
    \item International: \url{https://www.firstlegoleague.org}
    \item Local edition of the FLL: \url{https://www.firstlegoleague.udl.cat}
  \end{itemize}

  \vfill
  \begin{itemize}
    \item Check two documents available at the CV (Resources/Project)
      \begin{itemize}
        \item \texttt{APE\_FLL\_overview.PDF}
        \item \texttt{APE\_FLL\_what\_the\_client\_wants\_more\_or\_less.PDF}
      \end{itemize}
  \end{itemize}
\end{frame}

\subsection{Task 1: Create Logic Model (UML Class Diagram)}
\begin{frame}
  \frametitle{Block B - Domain | Task 1: Create the Logic Design of FLL}
  The logic or conceptual design refers to the database of the system → classes, associations, primary and foreign keys...
  \vfill
  Your first task is to create the UML Class Diagram of the FLL as it is described in \texttt{APE\_FLL\_overview.PDF}.
  \vfill
  \begin{exampleblock}{Modeling tools}
    draw.io, Microsoft Visio, Lucid Chart, FigJam…
  \end{exampleblock}
\end{frame}

\subsection{Task 2: Create Table of Stakeholders}
\begin{frame}
  \frametitle{Block B - Domain | Task 2: Create a table of stakeholders}
  Prompt:
  \textit{"Within the context of a software engineering project intended to the design and development of web-based system for the First Lego League, who are or might be regarded as stakeholders?"}
  \newline
  Copilot:
  "(…) Stakeholders include anyone who has an interest in or is affected by the system. Here’s a breakdown of potential stakeholders: students, coaches, mentors, judges, referees (…)"

  \vfill
  \begin{itemize}
    \item Complete the following table \textbf{(between 5 and 10 stakeholders)}
  \end{itemize}
\end{frame}

\begin{frame}
  \frametitle{Block B - Domain | Task 2: Create a table of stakeholders}
  \begin{tabular}{|p{0.35\textwidth}|p{0.60\textwidth}|}
    \hline
    \textbf{Stakeholders} & \textbf{Overview} \\
    \hline
    Name. \newline Primary or secondary? & What is its relationship with the system? \\
    \hline
    & \\
    \hline
    & \\
    \hline
    & \\
    \hline
    & \\
    \hline
    & \\
    \hline
    & \\
    \hline
    & \\
    \hline
  \end{tabular}
\end{frame}

\begin{frame}
  \frametitle{Block B - Domain | Submission}
  Please submit the UML diagram and table of stakeholders through the CV (look for Activity-1)
  \begin{itemize}
    \item ZIP file
    \item File name: \texttt{Group\_X\_NameStudent1\_NameStudent2.zip}
  \end{itemize}
  \vfill
  This activity will be evaluated by Sergio.
\end{frame}

\end{document}
